\documentclass[fontsize=11bp, paper=a4]{scrartcl}
\usepackage[utf8]{inputenc}

\usepackage[english,main=serbian]{babel}
\usepackage[left=1cm, right=1cm, top=2cm, bottom=2cm]{geometry}
\usepackage[automark]{scrlayer-scrpage}
\usepackage{amsmath}
\usepackage{mathabx}
\usepackage{hyperref}

\begin{document}

\ohead[]{}
\chead[]{Univerzitet u Kragujevcu}

\begin{center}
    {\Large Seminarski rad}

    \normalsize{Razvoj i oblasti veštačke inteligencije}
\end{center}

\hfill student: Željko Simić 3vi/2023

\hfill mentor: Tatjana Stojanović

\linespread{0.8}

\section{\normalsize{Uvod}}
Stremljenjem prema građenju  primenljivih mehanizama u zadatom okruženju koje nam je pristupačno, često upadamo u nevolje\cite{failbilism} - iskušenički ili iz radoznalosti izazovno što nas vodi ka svesnom ili podsvesnom skepticizmu.
Civilizacije\cite{humancpu} koje su sada uveliko prisutne u ovom svetu su bile građene i održavane pod odgovornost ljudske rase kroz sve rizike subjektivnog shvatanja. 
Sve to u nadi da se održi čistost, opreznost u raznim aspektima. 
Nužnim zlom dolazilo se do kombinacije objektivno ili apsurdnih ("\textit{de jure}")\cite{kuhn_vs_popper} ili plemenitih jedinstvenih zakona\cite{dostojevski}. 
Pomodarski-sujeverno propagiralo dalje u budućnost, pa su tu prisutni pojmovi : onotologije, gnosticizma, konformizma, itd. 
Sa druge strane, sve te kombinacije jedinstvenih su ličile jedna na drugu - ili uz slučajnost, ili plagiranjem, ili nasledstvom - epistemiološki, pragmatično, transcedentalno, itd.
 Svako sa inicijativom da rastumači svoje pristupe pred narodom susretao se sa izazovima da legitimitet tog iznesenog sagledanja bude održan pod posledičnošću raznolikih uzroka.
Kombinacijom spontanosti, a i nekih prethodnih običaja se dolazilo do saznanja - različito tumačenih (ne)potpuno. Tumačenja streme da budu univerzalno predstavljena, da bi se otklonio faktor zabune i premostila, što je očiglednije i opreznije moguće, među javnost i trajanje. 


Mističnost intrističkog bića prirode nam ne omogućava toliko jednostavnu prvobitnu moć da komuniciramo sa našim okruženjem sa našeg stanovišta opažanja uz pomoć naših dostupnih čula, pa smo s time se domogli raznih oruđa, tako i medijuma. 
Razvili smo konvencije - kodove, pod datim nazivima, sve je to moguće skupiti u jedan \textbf{jezik}. Za pojam jezika su se razvili srodni pojmovi za rastumačavanje tih jezika - gramatičkih, sintaksnih, semantičkih.
Da ne bi iste reči - kodove ponavljali u nekim kompleksnijim situacijama ljudi su generisali \textit{prečice} i formirali tako kodove jednostavnih sintaksi i složeno semantički. 
Gde ponovno morali da naprave dobru organizaciju da ne bi došlo do zabuna ili redudantnosti (koje su neizbežne, većinski prvobitno). Od sedam grčkih mudraca - među kojim se ističe trgovac Tales\cite{tales}, do savremenika moje shvatanje je da se stremi uveliko slepo, sa prividom čistote i plemenitosti ovog univezuma, pa da smo pretpostavljeni kritičkom razmišljanju, špekulisanju, prioritetizaciji evaluacija, \textit{kombinatornoj eksploziji} dodirljivim fenomenima, sve do noumenona. Izbegavanjem ponovnim uticajima poroka, već kakvoj takvoj trezvenosti i svojstvenosti.
Tako i niko pouzdan ne može dati neki krajnji formalni iskaz o sveopštem potpunom mehanizmu do dolaženja do svih varijanti saznanja - što dolazi do nekog vida \textit{pseudo-nauke}, tj. ponovnog isticanja privida krajnje plemenitosti \textit{Logos-a}\cite{Logos}, meritokratije, telepatije, samoodrživosti/individuacije\cite{jung}, utopije. Kako usputna opažanja čula se zapletu sa našim bićem, tako i pomisli - ideje se suoče sa našim uverenjima. 
Kako celebralni rad svesno ili nesvesno propagira impulse, tako i mi dolazimo do subjektivnih zaključaka misaono. 
Kako emocije su posledično zastupljene tim procesom, tako i umno osećamo. 
Tu se stvara kolotečina našeg prisustva u ovom okruženju, koliko nas dobro zdravlje posluži. 
Inspirisani tim naslednim mehanizmom živog bića, srljamo da napravimo mehanizam koji stremi njegovom oponašanju kroz \textit{a posteriori} razmatranja, razdvajanjem, i uprošćavanjem njihovih tumačenja da bi uhvatio svaki tren vremenskog prostora života. Ali da se poslužim pojmom potpunosti funkcionalnosti (NAND-kola)\cite{nand} kao neki mali korak koji je obeležio dosadašnje napredovanje (iz Arhimedove ``Eureke``\cite{eureka} u neku novu ``Eureku`` prema stremljenjima titana Prometeja oko kojih orbitiraju višebrojna književna dela) uz analogiju iz grčke mita o Arijadninom koncu i maču\cite{ariadna}.
\subsection{\normalsize{Pogled na pojmove inteligencije, znanja, ljudske imitacije}}
\subsubsection{\normalsize{Istorija temelja veštačke inteligencije}}

Inspirativna dela potiču od Aristotelovog zapažanja pojmova "\textit{forme}" po čoveku i "\textit{materije}" po materijalu bronze, vodio je ka apstrakciji podataka i simboličkom proračunu, da bi reprezentovao manipulaciju nad obrascima zarad dostizanja rešenja, što bi opet projektovalo sve u sadašnjost gde se po primeru koristi elektromagnetni materijal. Moć zaključivanja zarad sticanja forme silogizma je ukazivao tim instrumentom \textit{modus ponens} ($((p\implies q)\land p)\implies q$).
Tako i Francis Bekon u 1620. godini je rastumačio neke pojmove kombinovanjem Platonovog i Aristotelovog rada, pojam neophodnosti i dovoljnosti "\textit{svojstava}" kombinovao sa pojmom "\textit{forme}" sve se to svelo pod primenu u daljoj budućnosti 1981. za opis algoritama u veštačkoj inteligenciji, pa time rastumačio neke problematike NP-teških (ali ne i NP-kompletnih\cite{nondeterministic}, kao što je Tjuringov Halting problem) algoritama. 
Paskal 1642. je naveo javnost u praktičnom smislu da mehanizmi mogu dostići misaone kapacitete, nakon svih mehanizama skladnih za proračune, tj. oponašanje matematičkih operacija, praćenje vremena (antikitera, abakus, Napierove kostiju, sat Lazara Hilandarca\cite{lazar}, Šikardov sat). Lajbnic je bio oduševljen sa mognućnošću da se automatizuje logiku dokazivanja na osnovu pretpostavik tako stvorio Lajbnicov točak. 
Dekart je svojim opažanjem tumačio pojam dualnosti bića, pa sa time premostio algebarsko sagledanje geometrije svojim razvitkom analitičke geometrije. Dekart je samim tim težio usaglašenosti uma i tela.
Formalno opisivanje fizičkih pojava putem matematičkih pojmova ustanovljeno od strane Tomasa Hobsa 1651, uticalo je na današnjicu.

Po racionalnom i empirističkom sagledanju spoljašnjeg sveta tok priče oslanja se na prethodno pomenutu priču. 
Stranu manifestovanja racionalnog sagledanja daje se značaj Platonu i sledbenicima Paskalu, Dekartu i Lajbnicu, koji su brigu preusmerili na izgled tumačenja svojih sagledanja.
Primena ovakvog sagledanja stremi ka formiranju razumevanja svetskih pojava, sticanja čvrstog uverenja da je spoznat svaki obrazac. 
Sa druge strane, za manifest empirističko sagledanja daje se značaj Aristotelu, pa nadalje Hobsu, Džonu Loku, Hjumu, koji su brigu preusmerili ka opažanjima onom što su prosvedočili, pa na to stekli utisak-podsvesno/pomisao-svesno svojom slobodom mišljenja zarad asocijacija kao glavnom mehanizmu tumačenja. 
Primena ovog sagledanja u veštačkoj inteligenciji se ističe kroz \textit{memorijsku organizaciju semantičkih mreža}, \textit{MOPS} i razumevanja prirodnih jezika. 
Takođe je primenljivo u kognitivnoj psihologiji, sa stohastički-spontanim pristupima gde utiče na razvitak \textit{Bajesovske verovne mreže} (BBN - koja podstakla Tjuring-kompletno stohaističko analiziranje kao sistem). 

Imanuel Kant je tu kao manifest kombinovanja ovih dveju struja, gde i njegov uticaj se prožima kroz uvodni deo ovog seminarskog rada.

Razvitak formalne logike pri mehanizmima automatizacija označio se utvrđivanjem teorije grafova - Ojler i naziranjem teorije konačnih automata - Lajbnic.

Čarls Bebidževa \textit{diferencijalna mašina} za evaluaciju polinomijalnih funkcija je bila povod uvodjenja pojma računskog programiranja opšte-namene svojim \textit{analitičkim motorom}. 
Ada Lovlejs je podržala doprinoseći tom radu, pritom upoređujući moć analitičkog motora kako manipuliše sa algebarskim obrascima sa moć Žakardove mašine za tkanje. 
Ti obrasci su mogli biti enkodirani kroz kartice koje su bile uključene u rad mašine i predstavile pojam \textit{enkodiranja}. Gde je ovo neki vid oličenja kombinacije implementiranih racionalnih i empirijskih aspekata.

Formalni jezik predstavljen je kroz rad matematičara Džordža Bula uticajnim u formi veštačke inteligencije sa pažnom na formalizaciju zakona logike. 
Uvođenjem 3 logičke operacije ($\land$, $\lor$, $\lnot$), istinitonosnih vrednosti ($\top$, $\bot$) logičke analize. 
Delo Frege Gotloba je predstavljeno po nazivom \textit{predikatske logike prvog reda} u cilju da pokrije što više nedoumice Aristotelovog dela ``Logike``  s time što je uveo aksiomatsku osnovu "značenja" za iskaze podložne navođenju.
Što sa sobom uvodi predikatske simbole, teoriju funkcija, kvantifikovane promenljive čime se ustupa moć manipulisanja nad automatizovanim razumevanjem.
Vajthed i Rasel su zadali vrlo važno tumačenje zarad discipline veštačke inteligencije. 
Teoreme i dokaze bi posmatrali kao nisku karaktera, tako da se ne oslanjamo na intuiciju, ni na dobro tumačenje dokaza, već samim dobrim definisanjem zakona (sintaktičkih, uz to korišćenjem ili prethodno dokazanih zakona, ili "očiglednih") kroz te niske uz korišćenje formalnih pravila i logičkim sintaksama. To je i danas korišćen sistem za automatizaciju sistema pretrage dokaza teorema.
Alfred Tarski je dao osnovu veštačkoj inteligenciji \textit{teorijom reference}. Kombinovanjem rada Fregea, Rasela, Vajtheda dao je tumačenje metalogike, formalne semantike, teorije valuacije. Skot, Strači, Burstal, Plotkin su projektovali ovu teoriju na ideje programskih jezika i šire vezano za računarstvo. I tu mogu i da istaknem i oblast \textit{teoriju kategorija}\cite{category_theory} koja većinski daje sveopštu moć matematičkog izražavanja i uzor daljem razvijanju arhitekture računarstva.
%ovde bibliography

Jedan od osnova veštačke inteligencije koji ju je takođe procenjivao i podstakao nadalje jeste bio i usavršavanje digitalnih računara. Zadavanjem što pogodnije arhitekture arhitekture zarad sekvencijalnog izvršavanja (neformalno algoritamsko) operacija da se nazire na ideju biološke prirode modela inteligencije. Manifestuje se izrazito kroz delo Fon Nojmana. Tako da su se sve discipline premostile međusobno, tako isprepletale, sve u čast razvitka i proširenja ove oblasti u opštem smislu.

\subsubsection{\normalsize{Tjuringov test}}
Već pomenuti Alen Tjuring zaslužan doprinosom za svoje tumačenje pojma formalizma definicije algoritama. 
Ispoljio je nedoumice dvosmislenosti pojmova misli i mišljenja. Moglo bi se doći do nekih racionalnih odgovora, ali, eto, čovek je došao do ideje da uspostavi izražavanje kroz empirističko procenjivanje.
Kroz \textit{igru imitacije} inteligenciju mašine je procenjivao naspram ljudskog bića. 
U različitim oblicima zadataka je uzorkovao ponašanje algebarskih kalkulacija, glasovne komunikacije, itd.
Pritom je zasebno imao neku drugu osobu posrednika da proceni da li je, ili nije odgovor rezultat dela mašine ili čoveka. 
Što je više mašinin rad bio ocenjen kao delo čoveka ili izjednačeno dovoljno, za nju bi se moglo reći da je inteligentna blisko čoveku. 
Time se dao neki osnov sagledanja programa veštačke inteligencije. 

Može i biti inteligentnija ako bi nadmudrila čovekove sposobnosti pri obavljanju zadataka. Zašto bi je sputavali time da mora oponašati ljudske faktore kroz Tjuringov test na koja su Ford i Hajes dali svoja opažanja, uz sva uvažavanja.

Ada Lavlejs je ukazala na to da mašina će obavljati ono kako joj se nalaže, ali neće biti mudra, pa da pruži posao objektivno potrebne procedure (da validira neki sopstven rad, time nije u stanju da bude inteligentna) i to je Tjuring istakao, što se i odrazilo dalje na halting problem.
Expert sistemi, svojom sposobnosti vođenja dijagnostičkih analiza, pružaju mogućnost da dostave da dođu do nekih zaključaka nezapaženim od strane sopstvenih tvoraca, ali ne u potpunosti (tako se težilo nalaženjem načina kroz programske jezike, modele proizvodnih sistema, objektno-baziranim sistemima, reprezentacijama neuronskih mreža, itd.). Sve u stilu asinhrone dinamičnosti, kolekcija modularnih komponenata i skupina ponašanja koje se menjanju u zavisnosti od situacije. Ističe se Hilbert Simons sa svojim opažanjima da imunost bića zavisi od bogatsva okruženja u kom se zadesilo i učestalijim suočivanjima.

\subsubsection{\normalsize{Biološki i socijalni model inteligencije: Teorije agenata}}
Relativizam filozofije navodi se na osmatranje ograničenja (Witgenštajn ukazao na sumnjičavost prema naklonjenosti jeziku, običajima, društvu, nauci, misli apsurdno trivijalnim pitanjima) koja nas navode pasivi. Godel i Tjuring navode, tako slično, osumnjičavanju temeljima matematike. 
Ali pojedini istraživači suočili su se tim tvrdnjama racionalno-logilčki GOFAI-a (``Dobre stare veštačke inteligencije``) isticanjem situiranih/agent-baziranih modela ponašanja kao inspiracije. 

Stoga, došlo se do povoda da se mimikuje ljudski mozak zarad dosegnuća pojma inteligencije neuronskim mrežama. 
Genetski algoritam, slično, nastao po tvrdnjama da živa bića generacijski tako funkcionišu, zanemarujući ideju volje/principima/prognozama, ali i fantaziranja, već oslanjajući se na ideje apsolutnog praćenja same nužde ćudi - neurotično\cite{neuroza}. 
Tako slično ističu se primene i u socijalnim aspektima (npr. prognoza trgovina, berza, anketiranja).

Navode se 2 slučaja društvenih sagledanja. Prvi, pogled utopljen u kulturu, posledičnost, uzbunjivanje. 
Preostali, \textit{agent} na autonomno pojedinačnom nivou unutar kolektivnih raznovsnih ponašanja najprostijom interakcijom suočenih ostalih individua ili agenata. Kombinacija agent-orientisanog i uzbenjičkog sagledanja inteligencije podržavaju : \textit{(delimično) autonomne agente} - rešavaju zasebne delove problema, ili zasebno ili timski; \textit{situirane agente} - osuđene na nenadgledanu stohastičku obradu; \textit{interagujuće agente} - obradu vrši vođen nekim heuristikama; \textit{agenti struktuiranog društva} - vrše teritorijalno odvojeno poslove, sa rezultatom izrazito kooperativnog rada; \textit{fenomen inteligencije uzbenjičkog okruženja} - ističe se sveopšti doprinos kooperaciji, ne obazirući se na pojedinačni uticaj. Više vrsta agenata postoje: bubački - nepromisleno površno učenje, koordinacijski - podržavaju međusobnu interakciju agenata, pretraživačkih - za formiranje generalizacija i koncepata, odlučivačkih - dolaženja do zaključaka  limitranih obrada i informacija.
Društva se generišu zarad strukture tumačenja, strategije pretrage alternativnih rešenja, kreacije arhitektura za podršku interakcije agenata.
\section{\normalsize{Pregled oblasti primena veštačke inteligencije}}
\subsection{\normalsize{Igranje igara}}
\textit{Pretragom prostora stanja} moglo se doći do istraživanjima uz pomoć tradicionalnih društvenih igara (šah, mice, itd.) uz koja se moglo, isto tako, doći do pokrivanja njihovih zakonitosti formalizmima, umesto dvosmislenosti, suptilnosti, haotičnosti tumačenja. S obzirom, na glomaznost količine sadržaja prostora stanja pretrage, nahodi se da se primenjuje pojam heuristika kao nekih (ne)pouzdanih rano ustupljenih prečica. Stremi se i što boljoj efektivnosti i optimalnosti istih, a i isto determinističnosti upotupunjenosti prostora stanja (igra mice, 2007). 

\subsection{\normalsize{Automatsko razumevanje i pretraga dokaza teorema}}
Ističe se kao najstariji ogranak veštačke inteligencije, poteklog od dela Njuvela i Simona, preko Rasela i Vajtheda sa nahodjenjima da smatraju za matematiku kao neku formalizaciju izvedenu preko utemeljenih iniciranih aksioma, pa do spisa Bebidža i Lajbnica. Pretraga dokaza teorema je zaslužna za formalno tumačenje algoritama pretrage, formalno reprezentovanje jezika kao što su predikatska analiza i deklarativno-logičko programiranje - npr. Prolog (navedenim termovima, činjenicama, pravilima, itd.). Sve to pruža što upečatljiviju striktnost automatske obrade. Bez ikakvih heuristika svojom moći je istaklo neodlučno beskonačno mnogo dokazivih teorema, lema, hipoteza. Zaslužnost je ovoga za razvitak ekspertnih sistema. Primena mu je podstakla verifikovanje i dizajn logičkih kola, verifikovanje računarskih programa, kotrolu kompleksnih sistema. Heuristiku poboljšava ta priroda formalnog sintaktičko-logičkog izražavanja, pa projektuje se na rešavanje uviđene većine problema čovečanstva. Ali takođe, ljudi imaju priliku u nekim alternativnim primenama da daju svoj uticaj, ali sa mog subjektivnog stanovišta čovek bi trebalo da se smatra glavnim učesnikom, a mašina kao više uposlenijim asistentom (bez obzira na njene intelektualne kapacitete)\cite{zavisnost}, nasuprot Bojerovih, Murovih, Vorofovih, Spinksovih dobro produbljenih tumačenja.

\subsection{\normalsize{Ekspert sistemi}}
Domen discipline neke profesionalne-poslovne delatnosti u teoretskom i praktičnom smislu odazire se kroz iskustvo osobe, time umeće rešavanja problema može se posmatrati predrasudama heuristički i pre nego što je dođeno do rezultata što olakšava dosta posao. 
Sa tim aspektima ostvaren mehanizam koji omogućava takvim osobama da grade bazu znanja pokrivanjem ispoljavanjem svojih subjektivnih iskustava ponovnom interakcijom.
Primer mehanizma ovakvog pristupa jeste Stanfordov DENDRAL (namenjen za opažanje hemijskih reakcija, veza po vidu molekula) zarad razjašnjavanja problema. Još jedan primer, naslednik prethodnog, MYCIN je primenjen u medicini. Takođe, pristutni su PROSPECTOR (namena geoloških istraživanja), INTERNIST (namena internoj medicini), Dipmeter Advisor (istraživanje, beležejnje naftnih bušotina), XCON (namena za konfirguraciju naspram arhitekture VAX računara), a nadalje je prisutna IAAI (Inovativne Primene Veštačke Inteligencije) koja daje pažnju ovakvim sistemima. Svakako ne treba precenjivati ove mehanizme (mogu biti previše udubljeni, nepromišljeno-nefleksibilni, površno-nepotpuni u tumačenjima, poteži za procenu njihove doslednosti, tvrdoglavi).

\subsection{\normalsize{Razumevanje i semantika prirodnih jezika}}
Stremnja da jezik govora ljudi bude automatizovano prepoznavanjem obuhvaćen svojim obrascima, sa svim neintuitivnim njegovim preprekama (dvosmislenostima, lirskim ili neformalnim uzrečicama, stilskim figurama, gramatikama,...). Primer ovog ogranka je Vinogradov SHRDLU sistem - čiji je domen (mikrosvet) bio oblika, figura (``blokova``) i boja kojim se manipulisalo upitima. Univerzalnost tehnika se navodi da je delovano kroz proširenje \textit{semantičkih mreža}.
% treba bibliography
Tu sad može da se istakne savremen primer sistema \textit{generativnog pre-treniranog transformera} nastalog 2017, nastalim preko koncepta veštačkih neuronskih mreža \textit{ogromnog jezičkog modela} (LLM).\cite{gpt}

\subsection{\normalsize{Modeliranje ljudskog ophođenja}}
Težnja da se nadgleda kognitivni aspekt osobe, gde za razliku od ekspert sistema, ovi prate mentalno stanje ljudskog učesnika. Vrše se osude, pri pristupima učesnika tokom nekih aktivnosti, a i upotrebi verbalnih izražavanja, toka priče - doslednosti upotrebe ljudskog govornog jezika. Umesto da se a priori nameće teorija o stanju ljudskog uma, gde je podložnost odbačenosti nekih neopaženih aspekata moguća - po metodologiji behaviorizma, više se teži se ka upotrebi a posteriori računarstva zarad produbljenijeg rastumačivanja uzorkovanog učesnika. Psiholozi time imaju priliku da reimplementiraju ideje što formalnije po Lugeru.

\subsection{\normalsize{Planiranje i robotika}}
Nahođenje da se dizajn robota ostvaruje sa pažnjom na pokrete (kao jedinstvene akcije) njegovih komponenti u zadatoj promenljivoj sredini u kojoj su uključene i prepreke. 
Tako da namerena planiranja ne dovedu do rasejanosti. 
Postavlja se tvrdnja da ljudska bića koriste metodologiju hijerarhijske dekompozicije problema - isparčavanju i uređivanju redosleda prioritetnih akcija koje sleduju jedna posle druge. 
Ovaj proces može da padne da bude poteži računarskim sistemima - odluka kakva će biti heuristika uz sticanje percepcije dostupnim senzorima robota. 
Delimično autonomno spoznavanje više agenata može generisati adekvatnije viđenje prostora za koordinisanje. 

\subsection{\normalsize{Jezik i sredine namenjene veštačkoj inteligenciji}}
Moć skupina programerskih metodologija prouzrokovala je olakšanu implementaciju za različite namene (da li dinamičnije, da li produbljenije obrade) tehnika struktuiranja znanja zarad razvoja softvera. 
Programska oruženja odlikuju se kroz raznovsne programske paradigme dostizanjem efikasnosti prostorne i vremenske kompleksnosti. Premošćavanje sa ovog aspekta (primeri su C++, Java, itd.) na aspekt veštačke inteligencije (algoritama) deli u srži njihovih implementacija. 
Teži se daljim reprezentacijama kroz Prolog, Lisp (deklarativne programske jezike), Javu.
 Ističe se i zabrinutost oko preterane pomodarske pristrasnosti isključivo posebnim alatima.

\subsection{\normalsize{Mašinsko učenje}}
Obuka kao izazovna, ali neizbežna strana kreiranja inteligentnog ponašanja podstaknuta je aktivnim popravljanjem povratnom informisanošću ukoliko je potrebno da deluje inteligentnije. Program nije striktne nepromenjene implementacije kao uobičajenim situacijama. Lenatov \textit{Automatizovan Matematičar} je bio prvi poduhvat za otkrivanje matematičkih zakona podupregnutim Zemerlo-Frankel-Koši aksiomama teorije skupova. Koton je dizajnirao automatizovano generisanje ``zanimljivih`` celobrojnih sekvenci, nalik onim sinaktičkim izražavanjima Vajtheda i Rasela. Pomenuta upitivanja mehanizma u vezi skupa primera blokova zarad ostvarivanja pod Vinstonovim uticajem otkrivanja ``luka''. Ističe se ID3 uspešno primenjenim pri dokazivanju generalnih obrazaca iz primera. \textit{Meta-DENDRAL} uči pod usložavanjem poznatih sturktura zarad tumačenja organske hemije. 
\textit{Telresias} služi kao ``frontend'' sloj ekspertnih sistema, pretvaranjem saveta visokog nivoa u nove zakone baze znanja. 
{Hacker} razlaže planove zarad blokova performansa manipulacije sveta kroz iterativni proces razlaganja planova, testiranja, prepravljanja omašaja prethodno uočenih u potencijalnim planovima.

\subsection{\normalsize{Alternativne reprezentacije: Neuronske mreže i genetski algoritmi}}
Ističe se da zarad implementiranja što bolje pretrage i usavršavanja inteligentnih sistema podsticanje modela koji vrši paralelizovanje struktura neurona ljudskog mozga ili generacijskih obrazaca iznetim genetskim algoritmom i veštačkog života. Šema neurona sadrži ćelijsko telo koje se grana na više početaka - dendrita, jedno granu - aksona. Dendriti primaju signale ostalih neurona, kombinovanim impulsima prelazi se prag osetljivosti njihovom jačinom, odskok, time propagira dalje impuls kroz akson. Završeci sa ostalim dendritima neorona formiraju sinapse. Sinapse mogu biti pobuđujuće i umirujuće - tako što signale skupljaju ili odvajaju zarad stizanja do neurona u ukupnom delovanju.
Značaj računa neuronskih modela pojedinačno dovodi do njegove distibuisanosti i paralelizaciji obrade jedinica mreže u odnosu na veze i težine pragova. Razlozi primene ovog mehanizma naspram drugih jeste taj da su programi veštačke inteligencije jako zamšeni, poremetljivi, skloni šumovima (ljudska bit je pogodna za suočavanje sa tim nedoumicama prilikom poklapanja pokrivanjem obrazaca).

Genetski algoritmi se zasnivaju nad regenerisanjem solucija problema naspram prethodnih. Operacija ukrštavanja i mutacija, daje se kao analogon žudnjama živih bića zarad ponovnog dostizanja ``kvaliteta``. Jedinstvenost kao jedinica neurona ili segmenta rešenja podložnih paralelizaciji. Hilis ispoljava da ljudsko saznanje ih brzinski podstiče, a računara usporava, baš zbog te pogodnosti paralelizacije tokom pretrage baze znanja. 

\subsection{\normalsize{Veštačka inteligencija i filozofije}}
Ističu se pojmovi opažanja pojmova inteligencije, znanja, transparentnosti, ponovnog rastumačivanja, istine, moći izvršavanja zadatka. Isticanja Aristotelovog dela. Inicijativi ponovnog dizajna i evaluacije. Utvrđivanja održivosti algoritama i modela inteligentnog ponašanja. Manifestu jačeg modela inteligencije uz fizičke simbole izvođenja hipotetičkih iskaza Njuvela i Simona.
Dolaži se do produbljenih filozofskih analiza u oblastima veštačke inteligencije. Kako garantovati da računar spoznaje izraze prirodnog jezika? Proizvodnja razumevanje jezika zahteva tumačenje simbola. Mehanizam za spoznavanje mora da pridoda posrednički smisao simbola konteksta. Dostavlja se upitnost pojma značenja, tumačenja, odgovornosti nužne tumačenjem. Sve to u čast potrebnosti da ljudska osoba učestvuje u podsticanju ovih mehanizama.

\section{\normalsize{Rezime}}
Naspram šireg spektra problema ipak se dala šansa inicijativi da pokrije aspekte umeća obrade rešavanja, planiranja i komunikacija. Ističu se pogodnosti: prepoznavanja obrazaca, uvrstavanjem prethodnih prognoza, zbrinutosti o nedostacima formalnih tumačenja, procenjivanja, zbrinutosti sintektičkog i semantičkog oblikovanja izraza, zbrinutosti optimalnosti i efektivnosti heuristika, generisanja baze znanja profesionalnih delatnosti, spoznanja meta-nivoa zarad pouzdane kontrole stategija rešavanja problema.

\newpage
\appendix % headings numbered with letters
\section*{Appendix} 

\begin{enumerate}
    \item [3.]{\textbf{Da li uticaj misli tradicionalnog zapadnog sveta obazirale su se nad pitanjem odnosa tela i uma kao: 
    udaljene međusobno interagujuće pojave, 
    da umom se definiše fizički proces 
    ili da je telesni aspekt je samo iluzija racionalnog uma? 
    Diskusiju o ovome preneti i na važnost ovog pitanja u vezi veštačke inteligencije.}}
    
    To bi trebalo da zavisi od slučaja do slučaja, tako da u prvi pogled zahteva obezbeđivanje faktora skladnih da bi postojanje bilo moguće, time bi se svest o volji ubrajala kao umni kraj interakcije, a telesni kraj zasnovan okidajućim opažajima.
    U drugom pogledu, dajem uviđanje da pomodarko-sujeverni aspekt da volja može da utiče na spoljašnji svet u apsolutnom smislu, može dovesti do smrtonosnih posledica - neuviđajne agresivnosti. Naravno, osoba može svojom voljom sabirati neka opažanja, poput subjektivnih parola da bi iz dana u dan stremila ka nekoj produktivnoj aktivnosti, formiranju disciplina.
    U trećem pogledu, subjektivna opažanja u tom kontekstu mogu dovesti do otkrića sopstvenih mana, preuveličavanja značaja (usputnog egzorcizma) zarad dostizanja optimalne koristi. Ali sa druge strane ne možemo napustiti sopstvenu materijalnu stranu života kao pojave, očajavajući u svoj rasejanosti zabludom anulirajući prioritete.
    
    Tako naspram moje pomenute opservacije veštačka inteligencija može vršiti spoznavanja sredine, imati neka zapažanja, namere, odluke, pouzdanja, a pritom argumentovati sve to u svetlu neizbežne egzistencijalne krize svog trenutnog stanja. U svoj sopstvenoj sirovosti veštačke pojave, nije podložna podlesti nekim ljudskim faktorima.

    \item[5.] {\textbf{Moja lična zahtevanja oko inteligencije računarkog softvera.}}
    
    Svojom sopstvenom adekvatnoću svoje implementacije i opremljenosti namenjene za nameren aspekt posla. Utvrđivanjem kvaliteta prethodim pokrivena pouzdanost paranojom tvoraca ili na svoju ruku, projektovanjem kroz što profinjeniju strukturu faza razvoja. Univerzalnost i proširljivost učestvovanjem pojava zarad ustanovljavanje ulaznih parametara. Mogućnost čuvanja tekućih stanja, zarad ostvarivanja nezavisnosti od jedinstvenih uticaja, već i prostranog moći sopstvene održivosti.



\end{enumerate}
\newpage

\begin{thebibliography}{9}
    % \bibitem{texbook}
    % Donald E. Knuth (1986) \emph{The \TeX{} Book}, Addison-Wesley Professional.
    
    % \bibitem{lamport94}
    % Leslie Lamport (1994) \emph{\LaTeX: a document preparation system}, Addison
    % Wesley, Massachusetts, 2nd ed.
    \bibitem{failbilism}
    \url{https://iep.utm.edu/fallibil/}

    \bibitem{humancpu}
    \url{https://en.wikipedia.org/wiki/Computer_(occupation)}

    \bibitem{kuhn_vs_popper}
    \url{https://en.wikipedia.org/wiki/Kuhn-Popper_debate}

    \bibitem{dostojevski}
    Dostojevski: San smešnog čoveka

    \bibitem{tales}
    \url{https://www.britannica.com/biography/Thales-of-Miletus}

    \bibitem{Logos}
    \url{https://www.britannica.com/topic/logos}
    \bibitem{jung}
    \url{https://en.wikipedia.org/wiki/Individuation}
    
    \bibitem{nand}
    \url{https://en.wikipedia.org/wiki/Functional_completeness}

    \bibitem{eureka}
    \url{https://www.britannica.com/biography/Archimedes}
    
    \bibitem{ariadna}
    \url{https://en.wikipedia.org/wiki/Ariadne#Minos_and_Theseus}

    \bibitem{nondeterministic}
    \url{https://en.wikipedia.org/wiki/NP-hardness}
    
    \bibitem{lazar}
    \url{https://en.wikipedia.org/wiki/Lazar_the_Serb}
    
    \bibitem{category_theory}
    \url{https://plato.stanford.edu/entries/category-theory/}

    \bibitem{neuroza}
    \url{https://www.britannica.com/science/neurosis}

    \bibitem{zavisnost}
    \url{https://www.britannica.com/science/addiction}

    \bibitem{gpt}
    \url{https://en.wikipedia.org/wiki/Generative_pre-trained_transformer}
\end{thebibliography}

% \title{Seminarski rad}
% \subtitle{Razvoj i oblasti veštačke inteligencije}
% \date{}
% \author{}

% ja ne razumem nista

% {Seminarski rad} \hfill ovo radi \hfill student: Željko Simić

% \hfill what's up?
% \maketitle


% eyo


\end{document}